\documentclass{scrartcl}
\usepackage{ngerman}
\usepackage{url}
\usepackage{graphicx}
\usepackage{float}
\usepackage{placeins}

\title{Masterprojekt: Ähnlichkeitssuche in Multimedia-Daten}
\subtitle{Projektbericht}
\author{Christian Brzeski \and Jean Pierre Bineyti\and Andre Simon}
\date{\today}

\begin{document}
\maketitle

\section{Aufgabenstellung}
Als Alternative zur Tag-basierten Suche haben wir uns im Masterporjekt mit der Ähnlichkeitssuche zwischen Multimediadaten beschäftigt. Bei dieser instanzbasierte Suche haben wir uns auf Fotos speziell auf Fotos im jpg.- Format aus dem Internet fokussiert. 

\section{Implementierung}

\subsection{Vorbereitung}

Nachdem wir uns in unseren Teams eingefunden und untereinander ausgetauscht haben, haben wir mit der ersten Verteilung von Arbeitspaketen und mit der Einrichtung einer passenden Projektumgebung. Als Versionsverwaltung und Plattform für kollaboratives Arbeiten haben wir uns für Git entschieden und als Programmiersprache haben wir uns auf Java beschränkt, da es hierfür zahlreiche Frameworks im Bereich Computer Vision gibt und wir alle mit der Sprache gut vertraut sind. 
\\

Zunächst haben wir drei verschiedene Frameworks genauer untersucht, wobei jedes Gruppenmitglied eins dieser Frameworks bei sich auf dem PC installiert und anschließend getestet hat. 
\\

Hier ein paar Eigenschaften zu \textit{Lire} ...

Hier ein paar Eigenschaften zu \textit{OpenImaj} ...

Hier ein paar Eigenschaften zu \textit{OpenCV} ...
\\
\\
Aus dem Vergleich der drei gegebenen Frameworks haben wir uns schließlich auf OpenCV als unser gemeinsames Tool geeinigt, da es eine Vielzahl an Algorithmen und Methoden abdeckt und eine gute Doku sowie eine Reihe an guten Tutorials hat.
\\


Wir haben uns dazu entschieden zwei Versionen von OpenCV als Klassenbibliotheken aufzunehmen, um die Vorteile von OpenCV 2.4.11 und 4.1.0 jeweils nutzen zu können.

\subsection{Keypoint Detection/Description}

\subsection{Signaturen}

\subsection{Ähnlichkeitsmaß}

\subsection{Anfrageform}

\subsection{Schlussfolgerung}



\section{Ergebnisse}

\subsection Schlussfolgerung

\FloatBarrier

\section{Ausblick}

\section{Arbeitsaufteilung}

\end{document}
